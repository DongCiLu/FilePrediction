\section{Related Work}
\label{sec:relatedwork}

In this section, we provide a brief introduction on several existing works
which are related to our paper. We classify these existing works into two
categories. The first category consists of existing works on data placement
algorithms for distributed storage systems, while the second consists of those
on hybrid storage systems which aim to leverage SSD drives to improve data
access performance.

As large-scale distributed storage systems have been extensively used in HPC
area, the problem of distributing several petabytes of data among hundreds or
thousands of storage devices becomes more and more critical. To address this
problem, many data placement algorithms have been proposed. For instance,
Distributed Hash Tables (DHTs) have been used to place and locate data objects
in P2P systems \cite{Stoica2001, Ratnasamy2001, Cai2004}. Another replica
placement scheme called chain placement was also proposed and applied to some
P2P and LAN storage systems \cite{Rowstron2001, Lee1996, MacCormick2004}.
Honicky and Miller presented a family of algorithms named RUSH
\cite{honicky04} which utilizes a mapping function to evenly map replicated
objects to a scalable collection of storage devices and meanwhile supports the
efficient addition and removal of weighted devices.

To resolve the reliability and replication issues RUSH algorithms suffered in
practice usage, Weil et al. proposed a scalable pseudo-random data
distribution algorithm named CRUSH \cite{Weil2006} upon which our algorithm is
based. Besides optimally distributing data to available resources and
efficiently reorganizing data after adding or removing storage devices, CRUSH
exploits flexible constrains on replica placement to maximize the data safety
in the situation of hardware failures. CRUSH algorithm allows the
administrator to assign different weights to storage devices so that
administrator can control the relative amount of data each device is
responsible for storing. However, the device weights used in CRUSH algorithm
only reflect the capacities of storage devices, which means CRUSH algorithm
might not be effective anymore for hybrid storage systems consists of both SSD
and HDD devices since these two kinds of storage devices have totally
different performance characteristics.

Existing works on hybrid storage systems usually concentrate on two major
directions: either using a small amount of SSD-based storage as a cache
between main memory and hard disk drives or integrating SSD and HDD together
to form a hybrid storage system. Because of SSD devices' high cost and small
capacity, the cache-based solution is understandable and common. For example,
Srinivasan et al. designed a block-level cache named Flashcache
\cite{Flashcache} between DRAM and hard disks using SSD devices. Zhang et al.
proposed iTransformer \cite{Zhang2012} which exploits a small SSD to schedule
requests for the data on disks so that high disk efficiency can be achieved.
SieveStore \cite{Pritchett2010} adopts a selective caching approach in which
the access count of each block is tracked and the most popular block is cached
in SSD device.

Recently, efforts have been made to combine SSD and HDD together to construct
a hybrid storage system in which SSD is used as a storage device same as HDD
rather than a cache. Chen et al. designed and implemented a high performance
hybrid storage system named Hystor \cite{Chen2011}, which identifies data
blocks that either can result in long latencies or are semantically critical
on hard disks and store them in SSDs for future accesses. In order to reduce
random write traffic to SSD, Ren et al. proposed I\_CASH \cite{Yang2011} which
exploits the spacial locality of data access and only store seldom-changed
reference data blocks on SSD. Besides, ComboDrive \cite{Payer2009}
concatenates SSD and HDD into one address space via a hardware-based solution
so that certain data on HDD can be moved into the faster SSD space.

There are two main differences between existing works on hybrid storage
systems and our approach: First, most existing works on hybrid storage systems
only consider how to improve the utilization of SSD devices while they ignore
the reliability and replication issues that data placement algorithms, such as
CRUSH, try to resolve in HPC environment; Second, our approach tries to
monitor the usage of data blocks and store performance critical data on SSD
devices which is kind of similar to several existing works, like
\cite{Chen2011, Luo2012}, but beyond this, our approach applies machine
learning algorithm to predict the future usage of data blocks and store
potential critical data on SSD in advance.
